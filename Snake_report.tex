\documentclass{article}
\usepackage{graphicx} % Required for inserting images
\usepackage{ctex}
\usepackage{authblk}
\usepackage{abstract}
\usepackage{titling}
\usepackage{multicol}
\usepackage{geometry}
%\usepackage[numbers,sort&compress]{natbib}
\usepackage{gbt7714} 
\usepackage{amsmath}

\usepackage{hologo} %中文参考文献 解决中文名字之间用and连接的问题。

%\usepackage[super]{gbt7714} %中文参考文献
\geometry{headsep=0.5cm}
\geometry{left=25mm,right=25mm,top=28mm,bottom=30mm} %设置页边距
%-------------设置页眉页脚---------------
\usepackage{fancyhdr}
\pagestyle{fancy}
\fancyhf{}
\chead{{\zihao{-5}\songti Snake Project Report}}
\rhead{\thepage}           %该页码信息
\renewcommand{\headrulewidth}{0.5pt}  % 设置页眉线粗
\fancypagestyle{plain}{\pagestyle{fancy}}
\fancypagestyle{mystyle}{
    \chead{{\zihao{-5}\songti 贪吃蛇项目报告}\par\noindent }
    \rhead{\thepage}
    \lfoot{\zihao{-5} \heiti 作者简介:\songti 周德霄,男;\heiti 研究方向:\songti 医疗技术;E-mail:dexiao.zhou@fau.de}
    \renewcommand{\headrulewidth}{0.5pt}
    \renewcommand{\footrulewidth}{0.5pt}  % 设置页脚线粗
    }
%\pagestyle{empty}  % 首页之后清空格式
%------------设置页眉页脚------------------
%------------设置title和参考文献——————————————————
\title{\zihao{2}\songti \textbf{Java贪吃蛇项目报告}}
\author{\zihao{-4}\kaishu 周德霄\\ \zihao{-4}\kaishu 李安楠}
\affil{\zihao{5}\kaishu 埃尔朗根-纽伦堡大学,德国 \ 埃尔朗根91052}
\affil{\zihao{5}\kaishu 奔曜科技有限公司,中国 \ 上海}
\date{}
\ctexset{bibname=\zihao{5} 参考文献:} %重新定义参考文献的文本(这里加了冒号,并在最后排列成左对齐的样式,为了与中大学报的格式一致)
%-----------设置title和参考文献——————————————————————
\begin{document}

\maketitle
\thispagestyle{mystyle}
\ziju{0.085} %字间距
\noindent
\zihao{-5}\songti \textbf{摘 \ \ 要}:这里写摘要。

\noindent
\textbf{关键词}:模板;中山大学;空间仪器;关键词
\setlength\columnsep{0.8cm} %双栏间距长度
\begin{multicols}{2} %这里控制双栏,如果你需要使用单栏,将2改为1即可


\section{\zihao{4}\kaishu 引 \ 言}
\zihao{5}\songti 这是我在去年空间仪器总体设计上写的大作业的模板。形式上做到与中大学报的格式一致。该模板只供写作业使用,不是官方指定的模板,并且在一些小细节上可能与中大学报不完全一致,但应付作业足矣。 这是一个引用示范\cite{1998Gravitational}


\section{\zihao{4}\kaishu 大标题}
你可以通过section指令创建一个大标题


\section{\zihao{4}\kaishu 又是一个大标题}
如果你需要用到小标题可以使用subtitle指令。

\subsection{\zihao{-4}\songti 这是一个小标题}
小标题下字体与正文保持一致。

\section{\zihao{4}\kaishu 建议}
我非常建议你在需要创建新的小标题或大标题时直接复制我编辑好的代码然后再修改其中的文本变成你的新标题,这会保证你创建的标题的字号与字体与中大学报的要求一致。
论文的引用使用bibtex,将论文按bibtex的格式导入进ref.bib文件,在文章中利用cite指令即可引用。
\end{multicols} %控制双栏结束,如果像插入只占用一栏的图片,重新开启一个控制栏的命令

\bibliography{ref.bib} %引用文件名



\end{document}
